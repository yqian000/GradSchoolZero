\documentclass[11pt]{article}
\usepackage[utf8]{inputenc}
\linespread{1.25}

% add LaTeX packages to use here
\usepackage{amsmath}
\usepackage{amssymb}
\usepackage{amsfonts}
\usepackage{amsthm}
\usepackage{fancyhdr}
\usepackage{lastpage}
\usepackage{enumitem}
\usepackage{framed}
\usepackage[most]{tcolorbox}
\usepackage{geometry}
\usepackage{graphicx}
\usepackage{enumitem}
\usepackage{nicefrac}
\usepackage[light,noDcommand,uprightRoman]{kpfonts}

\DeclareSymbolFont{AMSb}{U}{msb}{m}{n}
%\DeclareMathSymbol{\R}{\mathbin}{AMSb}{"52}

 % set dimensions for page layout
\geometry
{
 left=5em,
 right=5em,
 bottom=5em,
 top=6em,
 headheight=110pt,
 showframe=false
}

\usepackage{hyperref}
\hypersetup{
    colorlinks=true,
    linkcolor=blue,
    filecolor=magenta,      
    urlcolor=blue!40!black!90!,
}


\setlist[itemize]{leftmargin=*} % prevents indenting of itemize
\newlist{myitemize}{itemize}{3}
\setlist[myitemize,1]{label=\textbullet,leftmargin=*}
\setlist[myitemize,2]{label=$\rightarrow$,leftmargin=4em}
\setlist[myitemize,3]{label=$\diamond$}

% abbreviations for some common math symbols
\newcommand{\Rset}{\hbox{$\mathbb R$}}
\newcommand{\Nset}{\hbox{$\mathbb N$}}
\newcommand{\Pset}{\hbox{$\mathbb{N}^{+}$}}
\newcommand{\Zset}{\hbox{$\mathbb Z$}}
\newcommand{\Qset}{\hbox{$\mathbb Q$}}
\newcommand{\Cset}{\hbox{$\mathbb C$}}

% theorem style
\newtheoremstyle{thmstyle}% name of the style to be used
  {0pt}% measure of space to leave above the theorem. E.g.: 3pt
  {0pt}% measure of space to leave below the theorem. E.g.: 3pt
  {}% name of font to use in the body of the theorem
  {}% measure of space to indent
  {\scshape}% name of head font
  {.}% punctuation between head and body
  {.5em }% space after theorem head; " " = normal inter-word space
  {}% Manually specify head

% theorem environment instance
\theoremstyle{thmstyle}
\newtheorem*{theorem}{Theorem}


% shaded and framed solution environment 
\makeatletter
\newenvironment{shadedSolutionBox}
  {\setlength{\OuterFrameSep}{0in}%
  \definecolor{shadecolor}{gray}{.8}% shading of shaded solution box
  \bigskip%
  \@nameuse{shaded*}\par\noindent\ignorespaces \textbf{Solution}.}
  {\hspace{\stretch{1}}\rule{1.5ex}{1.5ex}% adds filled box 
  \@nameuse{endshaded*}%
  \bigskip}
\makeatother

% alt shaded and framed proof environment 
\makeatletter
\newenvironment{shadedProofBox}
  {\setlength{\OuterFrameSep}{0in}%
  \definecolor{shadecolor}{gray}{.8}% shading of shaded solution box
  \bigskip%
  \@nameuse{shaded*}\par\noindent\ignorespaces \textit{Proof}.}
  {\hspace{\stretch{1}}\rule{1.5ex}{1.5ex}% adds filled box 
  \@nameuse{endshaded*}%
  \bigskip}
\makeatother

% shaded and framed theorem environment 
\makeatletter
\newenvironment{thm}
  {\setlength{\OuterFrameSep}{0in}%
  \definecolor{shadecolor}{gray}{1}% shading of shaded Theorem box
  \@nameuse{snugshade*}\par\noindent\ignorespaces%
   \@nameuse{theorem}}
  {\hspace{\stretch{1}}\scalebox{1.5}{\hbox{$\triangleleft$}}% adds triangle shape
  \@nameuse{endtheorem}%
  \@nameuse{endsnugshade*}%
  }
\makeatother


% header and footer elements of every page except the first.
\pagestyle{fancy}
\fancyfoot[L]{\textsc{CSc {\small\selectfont 30400}} }
\fancyhead[R]{{\small\selectfont\textsc{\studentLastName}}}
\fancyfoot[C]{{\small\selectfont\assignmentName}}
\fancyfoot[R]{{\small\selectfont\thepage\ of \pageref{LastPage}}}
\renewcommand{\headrulewidth}{0.8pt}
\renewcommand{\footrulewidth}{0.4pt}

% hline with variable thickness
\makeatletter
\def\thickhline{%
  \noalign{\ifnum0=`}\fi\hrule \@height \thickarrayrulewidth \futurelet
   \reserved@a\@xthickhline}
\def\@xthickhline{\ifx\reserved@a\thickhline
               \vskip\doublerulesep
               \vskip-\thickarrayrulewidth
             \fi
      \ifnum0=`{\fi}}
\makeatother

% length instance for \thickhline
\newlength{\thickarrayrulewidth} 
\setlength{\thickarrayrulewidth}{.8pt}

% header and footer for first page
\fancypagestyle{firstpage}
{
\fancyhf{}
\renewcommand{\footrulewidth}{0.4pt}
\renewcommand{\headrulewidth}{0pt}
\fancyhead[C]{%
\begin{tabular*}{\textwidth}{@{\extracolsep{\fill}}@{}l @{} c @{} r @{} }
{\small\selectfont\courseName}&{\normalsize\selectfont\assignmentName}&{\small\selectfont\studentFirstName\ \studentLastName}\\
\thickhline
&&{\scriptsize\selectfont\collaboratorNames}
\end{tabular*}%
}
\fancyfoot[R]{{\small\selectfont\thepage\ of \pageref{LastPage}}}
\fancyfoot[L]{{\footnotesize\selectfont\pdfcreationdate}}
}

\newcommand{\courseName}{Theoretical Computer Science} % course name





% your first name, your last name, and the assignment name
\newcommand{\studentLastName}{Lin} % your last name
\newcommand{\studentFirstName}{Chuantian} % your first name (and middle name, if applicable)
\newcommand{\assignmentName}{Assignment1} % the assignment name
\newcommand{\collaboratorNames}{[CT]. Lin} % if you worked with anyone to complete any part of the assignment, include the initial of the first name (and middle name, if applicable) and full last name of each of your collaborators, separated by commas (e.g., if you worked with Arthur Paul Pedersen and Sandra Lee, include "A.P. Pedersen, S. Lee")



\makeatletter
\newcommand{\vast}{\bBigg@{3}}
\newcommand{\Vast}{\bBigg@{4}}
\newcommand{\VVast}{\bBigg@{5}}
\newcommand{\vastl}{\mathopen\vast}
\newcommand{\vastm}{\mathrel\vast}
\newcommand{\vastr}{\mathclose\vast}
\newcommand{\Vastl}{\mathopen\Vast}
\newcommand{\Vastm}{\mathrel\Vast}
\newcommand{\Vastr}{\mathclose\Vast}
\newcommand{\VVastl}{\mathopen\VVast}
\newcommand{\VVastm}{\mathrel\VVast}
\newcommand{\VVastr}{\mathclose\VVast}
\makeatother


\begin{document} % marks the beginning of the document

\thispagestyle{firstpage} % institutes page style for first page

\setlength{\abovedisplayskip}{20pt} % space above math in align* environment
\setlength{\belowdisplayskip}{20pt} % space below math in align* environment




% marks the beginning of the document body


\begin{myitemize}  
\item[1.] Let $K$ be a subset of a topological space $X$.  Show that the following two statements are logically equivalent:
\begin{myitemize}\setlength{\itemsep}{1em}

    \item[(a)] For every family $\mathcal{F}$ of open sets in $X$:\\
    
   if $\displaystyle K\subseteq \bigcup_{O\in \mathcal{F}}O$, then there exists a finite subfamily    $\displaystyle\mathcal{F}_{0}\subseteq \mathcal{F}$ such that  $\displaystyle K\subseteq \bigcup_{O\in \mathcal{F}_{0}}O.$ 
    
    \item[(b)] Every family of closed subsets of $K$ enjoying the finite intersection property has a nonempty intersection.
\end{myitemize}

\item[2.] Prove that the image of a compact set under a continuous function is compact.

\item[3.] Consider a real-valued function $f$ defined on a convex subset $C$ of $\Rset^{n}$ such that for every $x_{1},x_{2}\in C$, there exists $\alpha\in (0,1)$ for which $f(\alpha x_{1} +(1-\alpha)x_{2})\leq \alpha f( x_{1} +(1-\alpha)f(x_{2})$.
\begin{myitemize}\setlength{\itemsep}{0.5em}
    \item[(a)] Show that the function $f$ is convex if it is continuous.
    \item[(b)] Does the converse of statement (a) also  obtain in general?  Justify your answer.
\end{myitemize}

\item[4.]  Let $\Sigma$ be an alphabet.  Prove that the family $\mathcal{P}\bigl(\Sigma^{*}\bigr)$ of all languages over alphabet $\Sigma$ is uncountable.

\begin{shadedSolutionBox} That the set $\Sigma^{*}$ of strings over $\Sigma$  is countable has been shown during lecture. Accordingly consider a surjection $\phi:\Nset\longrightarrow\Sigma^{*}$.  Define a mapping $\Phi:\mathcal{P}\bigl(\Nset\bigr)\longrightarrow\mathcal{P}\bigl(\Sigma^{*}\bigr)$ by setting for each $\nolinebreak{R\in \mathcal{P}\bigl(\Nset\bigr)}$:
\begin{align*}
    \Phi\bigl(\,R\,\bigr)\quad\coloneqq&\quad \Biggl\{\;w\in \Sigma^{*}\,:\,w=\phi(n)\mbox{ for some }n\in R\;\Biggr\}.
\end{align*}
That is, the value of $\Phi$ when applied to $R$ is the image of $R$ under $\phi$.  It is readily verified that $\Phi$ is a bijection.

For \emph{reductio ad absurdum}, assume that  $\mathcal{P}\bigl(\Sigma^{*}\bigr)$ is countable.  Then there is a surjection $\nolinebreak{\Psi:\Nset\longrightarrow\mathcal{P}\bigl(\Sigma^{*}\bigr)}$, whereby it follows that the composition $\Phi^{-1}\circ \Psi$ is a surjective mapping from $\Nset$ onto $\mathcal{P}\bigl(\Nset\bigr)$, the existence of which is impossible since $\mathcal{P}\bigl(\Nset\bigr)$ is uncountable, as demonstrated during lecture.
\end{shadedSolutionBox}
\end{myitemize}




\end{document}